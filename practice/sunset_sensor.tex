\chapter{Sunset Sensor}

In this lab assignment you will create a sunset sensor using a light-dependant resistor (LDR).
If there is plenty of light, it means than the Sun is high in the sky.
In the case of complete darkness, the Sun has already gone.
The sunset detector will display an alarm (either an LED or buzzer) for intermediate lighting ranges.

The detector consists of two parts that communicate wirelessly, namely the sensor board and the processing board.
The sensor board contains the LDR and an XBee that takes measures and sends them to the other part.
The processing board contains an XBee to receive the data and an Arduino to process it.
The processing board also contains the alarm (LED, buzzer or both).

Use a resistor in series with the LDR to obtain a range of values readable for the XBee analog input.
Take a sample every 200 ms.

On the processing board, use the API firmware to be able to serially read the values that the remote XBee is sending.
Check which is the received value and if it is in the range of interest (intermediate) activate the alarm (LED or buzzer).

If you are willing to do more complicated stuff, try to move the alarm to the sensor board. 
Now the sensor board receives the data, sends it to the processing board for processing, and waits for an instruction from the processing board to ring or light the alarm.
